\section{Introduction}

The presence of shadow zones and illumination variability is an extremely common occurrence in seismic images, irrespective of whether they are obtained by Kirchhoff migration, reverse time migration (RTM) \citep{baysal1983reverse}, or more generally least squares migration (LSM) \citep{nemeth1999least, ronen2000least, dai2012multi}. Of course the choice of the migration algorithm impacts the severity of the phenomenon --- in general the effects are the greatest for ray based methods, and gets progressively better as one moves to wave equation based methods, and wave equation based least squares migration methods. However none of these algorithms can completely eliminate the phenomenon. The most common cause of shadow zones and illumination variability is poor acquisition design that lead to an uneven fold of coverage, and the formation of acquisition holes in azimuths, offsets and angles. Additionally, intrinsic factors such as rapidly changing velocity profiles could also lead to this phenomenon, for example such effects are extremely common in areas with very complex salt body geometries.

Other than the choice of the migration algorithm, mitigation strategies for acquisition related illumination artifacts involve techniques like pre-processing the data to interpolate missing traces, and normalizing the trace amplitudes based on fold. However uneven (or no) illumination caused by complex velocity models is much harder to compensate using these simple methods, and one must use regularization schemes instead. These regularization strategies are most easily implemented in the context of LSM, and our focus in this abstract will be on their wave equation variants. The use of regularizers for LSM is not new --- it has been applied to remove the effect of acquisition footprints from the image due to missing data \citep{nemeth1999least}, and it can be additionally used to further suppress cross-talk and missing data artifacts in LSM imaging of blended data \citep{tang2009least, xue2015seismic}. In the context of subsalt imaging, where illumination compensation is particularly challenging, a typical idea is to perform LSM along an extended axis and regularize the image along it, as has been done for example using geophysical regularizers along the offset ray parameter axis (or equivalently angles), and using steering filters \citep{prucha2002subsalt, clapp2005imaging, clapp2005regularized, kuehl2001generalized}. It should be noted that even with perfect acquisition, in areas with rapidly changing velocities, shadow zones tend to be the most prominent in the high frequency regime where wave propagation is well approximated by high frequency asymptotics, i.e. ray theory.

Tomographic full waveform inversion (TFWI) \citep{biondi2012tomographic, biondi2013tomographic, TFWI.gp} was originally developed as a heuristic method to mitigate some of the effects of cycle skipping that adversely affect the convergence of full waveform inversion (FWI). It belongs to a general class of methods, all of which are based on the idea of model extension --- the most notable examples being extension in subsurface offset or angle \citep{sava2003angle, rickett2002offset}, source extension \citep{huang2015full, huang2016matched}, and extension in time lag as in the case of TFWI. While several different strategies have been suggested to speed up the convergence of TFWI, such as wavelength continuation and scale separation \citep{almomin2012tomographic, almomin2013tomographic}, it was recently noted that the method of variable projection could be applied to solve the TFWI problem \citep{barnier2018full}, within a broader scheme of methods called full waveform inversion by model extension (FWIME). In this approach, at every iteration one fixes the physical velocity model, and solves a linear least squares optimization problem to obtain a solution for the extended model that minimizes the TFWI objective function. We will refer to this problem as the \textit{``TFWI linear inverse problem''}.

In this abstract, we will show that the frequency domain formulation of the TFWI linear inverse problem is equivalent to regularized extended LSM (with a specific choice of regularizer) along the extended frequency axis, which is simply the Fourier dual of the extended time lag axis. Because of the form of the objective function, we will additionally demonstrate a certain form of equivalence (to be made precise in the next section) with non-extended LSM. Using simple numerical examples, we will demonstrate how this formalism can be used to remove the effect of illumination artifacts in migrated images or extended migrated images, that arise either due to the complexity of the velocity model or from acquisition holes.