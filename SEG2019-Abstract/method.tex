\section{Theory}

\vspace*{-0.2cm}
\subsection{Setup of the TFWI linear inverse problem}

\vspace*{-0.1cm}
The frequency domain formulation of the TFWI linear inverse problem and an efficient solution strategy to solve it has been developed in \cite{Sarkar.sep.172.rahul1}. The formulation leads to the problem of minimization of the linear least squares objective function $\hat{J}(\cexth)$ given by the expression

\vspace*{-0.7cm}
\begin{equation}
\sum_{k=1}^{N_\omega} \sum_{i=1}^{N_s} ||\Rsamp_i \uexth_{ki} - \residh_{ki}||_2^2 + \epsilon^2 \left\lVert\; \frac{\partial}{\partial \omega} \cexth \right\rVert^2_2 + \gamma^2 ||\cexth||^2_2\;,
\label{eq:Jtfwi-mod-freq}
\end{equation}
\vspace*{-0.5cm}

that needs to be solved at every iteration inside a variable projection scheme for a fixed value of the physical velocity model $\corig$, with the goal of obtaining $\cexth^{\ast}$ that minimizes \eqref{eq:Jtfwi-mod-freq}. The expression above is readily obtained from the original time domain TFWI objective function \citep{TFWI.gp}, with the only exception of the $||\cexth||_2$ term which has been added to make the linear inverse problem well-posed, by Fourier transforming all quantities in time and by using the fact that the Fourier transform preserves the $L^2$ norm. In expression \eqref{eq:Jtfwi-mod-freq}, $\epsilon, \gamma \in [0, \infty)$, $N_\omega$ and $N_s$ are the number of discrete frequencies and sources in the inversion respectively, and $\cexth$ is the frequency domain extended model. For the $i^{th}$ source and $k^{th}$ frequency, $\Rsamp_i$ denotes the restriction operator that samples the wave field at the receiver locations, $\residh_{ki} = \dorigh_{ki} - \Rsamp_i \uorigh_{ki}$ where $\dorigh_{ki}$ represents the recorded data, and $\uorigh_{ki}, \uexth_{ki}$ represents the physical wave field and the extended Born modeled wave field respectively, obtained by solving the following system of Helmholtz equations for angular frequency $\omega_k$ and source function $\srch_{ki}$:

\vspace*{-0.6cm}
\begin{equation}
\begin{split}
& \left( \mathbf{\nabla}^2 + \omega_k^2 / \corig^2 \right) \uorigh_{ki} = - \srch_{ki} \;,\\
& \left( \mathbf{\nabla}^2 + \omega_k^2 / \corig^2 \right) \uexth_{ki} = \left( 2 \omega_k^2 / \corig^3 \right) \uorigh_{ki} \cexth_{k} \;.
\end{split}
\label{eq:upde-helmholtz}
\end{equation}
\vspace*{-0.6cm}

In the second equation in \eqref{eq:upde-helmholtz}, $\cexth_{k}$ represents the component of $\cexth$ corresponding to the $k^{th}$ frequency. Discretizing \eqref{eq:Jtfwi-mod-freq} on a spatial grid leads to the linear least squares problem of minimizing the objective function (still denoted as $\hat{J}(\cexth)$)

\vspace*{-0.7cm}
\begin{equation}
\sum_{k=1}^{N_\omega} \sum_{i=1}^{N_s} ||\Rsamp_i \uexth_{ki} - \residh_{ki}||_2^2 + \epsilon^2 \left\lVert\; \D \cexth \right\rVert^2_2 + \gamma^2 ||\cexth||^2_2\;,
\label{eq:Jtfwi-freq-discrete}
\end{equation}
\vspace*{-0.5cm}

where all quantities now represent their discrete equivalents, and we take $\D$ to be the forward difference derivative operator acting along the frequency axis.

In the next subsections, we will show the connection of \eqref{eq:Jtfwi-freq-discrete} with regularized LSM and extended LSM, but for this we will assume that we are in the Born scattering regime, i.e. the physical velocity $\corig$ is a reasonably good approximation of the smooth background velocity model, so that $\residh_{ki}$ is well approximated by first order Born scattering. Notice that in this regime, if $\dmorig$ represents the perturbation in the physical velocity (independent of frequency), then the Born scattered wave field $\dvorigh_{ki}$, for the $i^{th}$ source and $k^{th}$ frequency, is given by (note the resemblance with the second equation in \eqref{eq:upde-helmholtz})

\vspace*{-0.4cm}
\begin{equation}
\left( \mathbf{\nabla}^2 + \omega_k^2 / \corig^2 \right) \dvorigh_{ki} = \left( 2 \omega_k^2 / \corig^3 \right) \uorigh_{ki} \dmorig \;.
\label{eq:born-helmholtz}
\end{equation}
\vspace*{-0.5cm}

\subsection{Relation to regularized extended LSM}

\vspace*{-0.1cm}
The general form of regularized extended LSM in the frequency extended space (the same form is valid in any extended domain) is given by the minimization of any objective function of the form

\vspace*{-0.6cm}
\begin{equation}
\Phi(\cexth) = \sum_{k=1}^{N_\omega} \sum_{i=1}^{N_s} ||\Rsamp_i \uexth_{ki} - \residh_{ki}||_2^2 + \epsilon^2 \left\lVert\; \A \cexth \right\rVert^2_2 \;,
\label{eq:Jextreg}
\end{equation}
\vspace*{-0.4cm}

where $\A$ is a linear operator. We will not consider non-linear regularization as they are not common in seismic imaging applications. Notice that $\uexth_{ki}$ depends linearly on $\cexth_k$ by \eqref{eq:upde-helmholtz}, and thus \eqref{eq:Jextreg} is indeed a linear least squares problem. The interpretation is simple --- the first term is the data fitting term that seeks to find $\cexth_k$ that minimizes the data misfit for the $k^{th}$ frequency, while the second term is the regularization term that constrains the relationship between the different $\cexth_k$'s. Thus \eqref{eq:Jtfwi-freq-discrete} can be put in the form of \eqref{eq:Jextreg} by simply noting that

\vspace*{-0.5cm}
\begin{equation}
\epsilon^2 \left\lVert\; \D \cexth \right\rVert^2_2 + \gamma^2 ||\cexth||^2_2 \; = \; \left| \left| \; 
\begin{bmatrix}
\epsilon \D \\
\gamma \I
\end{bmatrix}
\cexth \; \right| \right|_2^2 \;.
\label{eq:Jextreg_relation}
\end{equation}
\vspace*{-0.5cm}

\subsection{Relation to regularized LSM}

\vspace*{-0.1cm}
The relationship with regularized LSM is only slightly more difficult to establish. Similar to \eqref{eq:Jextreg}, the general form of regularized LSM is given by the minimization of any objective function of the form

\vspace*{-0.7cm}
\begin{equation}
\Psi(\dmorig) = \sum_{k=1}^{N_\omega} \sum_{i=1}^{N_s} ||\Rsamp_i \dvorigh_{ki} - \residh_{ki}||_2^2 + \eta^2 \left\lVert\; \A \dmorig \right\rVert^2_2 \;,
\label{eq:Jlsmreg}
\end{equation}
\vspace*{-0.5cm}

and the key difference compared to \eqref{eq:Jextreg} is that the minimization is now over the velocity perturbation $\dmorig$, which is purely in the physical space. Again by \eqref{eq:born-helmholtz} it is clear that \eqref{eq:Jlsmreg} represents a linear least squares problem. To establish the relationship with \eqref{eq:Jtfwi-freq-discrete}, let us first rewrite minimization of \eqref{eq:Jlsmreg} in the equivalent form

\vspace*{-1.0cm}
\begin{equation}
\begin{split}
& \text{minimize} \;\;\;\; \sum_{k=1}^{N_\omega} \sum_{i=1}^{N_s} ||\Rsamp_i \uexth_{ki} - \residh_{ki}||_2^2 + \frac{\eta^2}{N_\omega} \left\lVert\; (\I_{N_\omega} \otimes \A) \cexth \right\rVert^2_2 \\
& \text{subject to} \;\;\;\;\; \D \cexth = 0.
\end{split}
\label{eq:Jlsmreg-eqv}
\end{equation}
\vspace*{-0.7cm}

In the above expression $\otimes$ denotes the Kronecker product, $\I_{N_\omega}$ is the identity matrix of size $N_\omega \times N_\omega$, and the constraint simply expresses the condition $\cexth_{1} = \dots = \cexth_{N_\omega}$, as $\D$ is the forward difference derivative operator. One should also note that \eqref{eq:Jlsmreg-eqv} is equivalent to \eqref{eq:Jlsmreg} only because the second equation in \eqref{eq:upde-helmholtz}, and \eqref{eq:born-helmholtz} are functionally similar to one another, only differing in the right hand side ($\cexth_{k}$ vs $\dmorig$). It then follows from well known properties of the quadratic penalty method, that if one considers the family of unconstrained minimization problems (parameterized by $\epsilon$) with objective functions

\vspace*{-0.7cm}
\begin{equation}
\sum_{k=1}^{N_\omega} \sum_{i=1}^{N_s} ||\Rsamp_i \uexth_{ki} - \residh_{ki}||_2^2 + \epsilon^2 ||\D \cexth||_2^2 + \frac{\eta^2}{N_\omega} \left\lVert\; (\I_{N_\omega} \otimes \A) \cexth \right\rVert^2_2 \;,
\label{eq:Jlsmreg-eqv-uncons}
\end{equation}
\vspace*{-0.7cm}

such that $\epsilon \rightarrow \infty$, and $\pos_\epsilon$ is a global minimizer of \eqref{eq:Jlsmreg-eqv-uncons} at parameter value $\epsilon$, then if $\pos^{\ast}$ is any limit point of the sequence $\{ \pos_\epsilon \}$, $\pos^{\ast}$ is also a global minimizer of the constrained problem \eqref{eq:Jlsmreg-eqv} (see Theorem 17.1 in \cite{nocedal2006numerical}), and hence of \eqref{eq:Jlsmreg}. It is clear that \eqref{eq:Jtfwi-freq-discrete} is a special case of \eqref{eq:Jlsmreg-eqv-uncons} when $\A = \I$, and $\gamma = \eta / \sqrt{N_\omega}$. To make this work, in principle one should gradually increase $\epsilon$ and solve \eqref{eq:Jlsmreg-eqv-uncons} repeatedly, till the resulting minimizers converge. However this is not possible in practice as solving \eqref{eq:Jlsmreg-eqv-uncons} is expensive. An alternative strategy is to simply choose a large value of $\epsilon$ (which can be done with a few finite number of trials), and solve \eqref{eq:Jtfwi-freq-discrete} for this $\epsilon$ to get $\cexth^{\ast}$, and then get an estimate of the optimal $\dmorig^{\ast}$ by averaging

\vspace*{-0.5cm}
\begin{equation}
\dmorig^{\ast} = \frac{1}{N_\omega} \sum_{k=1}^{N_\omega} \cexth_{k}^{\ast} \;,
\label{eq:dmavg}
\end{equation}
\vspace*{-0.5cm}

where we have assumed that the inversion is performed over the same positive and negative frequencies, and so $\dmorig^{\ast}$ is real valued, because if $\cexth_{k}^{\ast}$ is the solution for any positive frequency, then $\overline{\cexth_{k}^{\ast}}$ is the solution for the corresponding negative frequency. In practice to reduce the computational cost by a factor of two, one can simply perform the inversion over the positive frequencies, and set the $\cexth^{\ast}_k$'s corresponding to the negative frequencies according to this relation. In what follows, we will refer to $\dmorig^{\ast}$ obtained this way to be the \textit{stacked inverted image}.