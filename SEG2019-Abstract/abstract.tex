\begin{abstract}

\vspace*{-0.6cm}
Many different factors such as sub-optimal acquisition design or rapidly changing velocity profiles can contribute to the formation of shadow zones in migrated images, or more generally in least squares migrated images. The solution of tomographic full waveform inversion (TFWI) using the variable projection technique leads to a linear least squares inverse problem. For a fixed background model, solving this problem is equivalent to performing regularized extended least squares migration. We show in this abstract that the frequency domain formulation of the linear inverse problem makes this connection even more apparent and straightforward. We then demonstrate that in the Born scattering regime, the solution to this problem can be used as a tool to compensate for the lack of illumination in shadow zones, and hence remove illumination artifacts from least squares migrated images, including the extended case. Our claims are backed up using 2D numerical examples, where in the first example we correct for illumination artifacts caused only by a low velocity anomaly, while in the second example we additionally remove the effect of acquisition holes. We thus discover an interesting application of TFWI in the regime of linearized waveform inversion, where the goal is the estimation of the reflectivity component only, and not the full velocity model.
\end{abstract}